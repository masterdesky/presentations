\documentclass[12pt, a4paper]{article}
\usepackage[margin=0.8in]{geometry}

\input{preamble.tex}
%TC:endignore

\title{AI learns the stellar spectrum}
\author{Balázs Pál$^{\,a,b}$%
    \thanks{E-mail: pal.balazs@ttk.elte.hu}
}
\affil{%
    $^{a}$Eötvös Loránd University, Department of Physics of Complex Systems \\
    $^{b}$Wigner Research Centre for Physics, Heavy-ion Physics Research Group
}
\date{July 27, 2023}

\begin{document}

\maketitle

\begin{abstract}
Examining the properties of stellar populations in and around the Milky Way is a crucial step towards the understanding of galactic evolution. Gaining insight into this process can provide us valuable information about the large-scale and long-term characteristics of both ordinary and dark matter. In this study we focused on the spectroscopic aspect of this investigation, by looking into how well autoencoder-based neural networks (AEs) perform in the processing and analysis of stellar spectra. We show that AEs are capable of learning the physical characteristics of stellar spectra to a considerable extent, even in noisy and low S/N conditions.
\end{abstract}

\end{document}
