\documentclass[12pt, a4paper]{article}
\usepackage[margin=0.8in]{geometry}

\usepackage[utf8]{inputenc}
\usepackage[english]{babel}
\usepackage[T1]{fontenc}
\usepackage{lmodern}

%----------------------------------------------------------------------------------------
%	MATH PACKAGES
%----------------------------------------------------------------------------------------

% Banish \phi from this realm
\renewcommand{\phi}{\varphi}

\usepackage{amsmath, amssymb, mathrsfs}
\usepackage{mathtools}

%----------------------------------------------------------------------------------------
%	DEFINING NEW FUNCTIONS
%----------------------------------------------------------------------------------------

% --------- MATH MODE ---------
% Equation numbering per section
%\numberwithin{equation}{section}

% \cdot instead of asterisk (*) symbol
\mathcode`\*="8000
{\catcode`\*\active\gdef*{\cdot}}

% --------- OTHER ---------

% tikz
\usepackage{tikz}

% Captions
\usepackage[font=scriptsize]{caption}

% Quotes
\usepackage[autostyle=false]{csquotes}
\newcommand{\q}[1]{„#1''} % Redefine quotations
%TC:endignore

\title{Learning stellar spectra with AI}
\author{Balázs Pál$^{\,a,b}$%
    \thanks{E-mail: pal.balazs@ttk.elte.hu}
}
\affil{%
    $^{a}$Eötvös Loránd University, Department of Physics of Complex Systems \\
    $^{b}$Wigner Research Centre for Physics, Heavy-ion Physics Research Group
}
\date{\today}

\begin{document}

\maketitle

\begin{abstract}
Examining the dynamics and spatial properties of individual stars and stellar populations in satellite galaxies surrounding the Milky Way through the lens of galactic archeology is crucial to understand galactic evolution. By gaining insight into this process, it can provide us valuable information about the large-scale and long-term characteristics of both ordinary matter and dark matter. Thorough studies usually require both photometric and spectroscopic observations and data that complementarily support each other. In this study we focused on the spectroscopic aspect of this problem, by investigating how well autoencoder-based neural networks (AEs) perform in the processing and analysis of stellar spectra. We show that AEs are capable of learning the characteristics of a stellar spectrum to a considerable extent, even in noisy and low S/N conditions. Moreover, they can be extremely valuable tools during the preprocessing stage of spectroscopic data analysis, where otherwise finding the black body continuum of an observed, noisy spectrum is challenging.
\end{abstract}

\end{document}
