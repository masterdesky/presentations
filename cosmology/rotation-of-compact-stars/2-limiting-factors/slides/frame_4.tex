%----------------------------------------------------------------------------------------
%	SLIDE 3.
%----------------------------------------------------------------------------------------
\begin{frame}
\frametitle{Kepler frequency in GR}

\begin{itemize}
	\item To only unknown in Eq. \eqref{eq:2} is the $V$ velocity of a fluid element along the equator. This can be determined by finding the extremum of
	\begin{block}{}
		\begin{equation} \label{eq:3}
			\mathrm{d}\tau
			=
			\int_{t_{1}}^{t_{2}} \mathrm{d}t\,
			\sqrt{
				e^{2 \nu} - r^{2} e^{2 \mu} \left( \Omega - \omega \left( R \right) \right)^{2}
			}
		\end{equation}
	\end{block}
	\item After solving the variational problem and expanding $V$, we get the solution as
	\begin{block}{}
		\begin{equation} \label{eq:4}
			V_{\pm}
			=
			\frac{R \omega'}{2 \psi'} e^{\mu - \nu}
			\pm
			\sqrt{
				\frac{\nu'}{\psi'}
				+
				\left(
					\frac{R \omega'}{2 \psi'} e^{\mu - \nu}
				\right)^{2}
			},
		\end{equation}
	\end{block}
	where $\psi' = \mu' + 1 / R$
	\item The physically interesting result is given by $V_{+}$, since that is the co-rotating case needed for Eq. \eqref{eq:2}.
\end{itemize}

\end{frame}