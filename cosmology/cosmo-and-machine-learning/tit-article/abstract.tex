\begin{abstract}
	A tudomány és a technika egymást gerjesztő és így exponenciálisan gyorsuló fejlődése gyökeresen formálta át életünk szinte minden szegletét az elmúlt évtizedekben. A tudományos módszertan is számtalan területen ment keresztül paradigmaváltáson, mely mögött első sorban a mért adatok és a felhasználható számítási kapacitás mennyiségének sokszoros nagyságrendbeli növekedése, valamint az adattárolási és adatfeldolgozási módszerek -- az előbbiekkel lépést tartani próbáló -- fejlődése állt. Vitathatatlan, hogy ezen fejlődés keretében a gépi tanulási módszerek az elmúlt több, mint 10 évben forradalmasították az ipar és a tudomány sok területét és egyelőre változatlan lendülettel kísérelnek folyamatosan újabbakat is meghódítani. A csillagászati és asztrofizikai kutatások különösen hosszú múltra tekinthetnek vissza ezen modern megoldások alkalmazása terén, ahol a különféle égboltfelmérő projektek már évtizedek óta hatalmas mennyiségű adatot termelnek, melyek kezelése így mindig is a legfrissebb eszközöket igényelte. Ugyanezen időszak alatt a kozmológiai tematikájú kutatások figyelme a kozmológiai standard modell, a $\Lambda$CDM irányába fordult. A technikai fejlődés a mérőműszereink pontosságának növekedését is egyben magával hordozta, amiknek köszönhetően kiderült, hogy a jelenleg elfogadott kozmológiai standard modell számtalan ellentmondást és pontatlanságot rejt magában. A kozmológiai kérdéseket vizsgáló felmérések adatainak minél hatékonyabb begyűjtése, valamint minél gyorsabb és pontosabb elemzése így az érdeklődés első számú célpontjává vált a modern csillagászatban. Jelenleg annak az átmenetnek lehetünk tanúja, ahogyan a számítógépes szimulációk, valamint a közeljövőben megkezdődő műszeres mérések gépi tanulással támogatott feldolgozása, hogyan formálja át alapvető ismereteinket a kozmológiáról és az univerzumról, amiben mindannyian élünk.
\end{abstract}