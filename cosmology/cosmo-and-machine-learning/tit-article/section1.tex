\section{Kozmológia...}
\q{\textbf{csgy könyv definíció ide, hogy mi a kozmológia}} áll a \textit{kozmológia} velős definíciója, Csaba György Gábor, \textbf{könyv címe} című csillagászati kisokosában. Persze kissé nagyképűségnek tűnhet (és igazából az is), de mégsem túlzás úgy tekinteni a kozmológiára, mint az emberiség legrégebbi tudományágára. Ásatási leletekből már régóta tudjuk, hogy az eleink érdeklődését már az ősidkben is felkeltették a körülöttünk található világ céljai, \textbf{its intentions} és az azt működtető fogaskerekek mibenléte. Honnan is jöttünk mi magunk és minden, ami körbevesz minket?

Habár ma már egyértelműen közelebb állunk ezeknek a kérdéseknek a megválaszolásához, mint évezredekkel ezelőtt, a pontos képről továbbra is csak a kényelmesnek nevezhetőnél bizonytalanabb lábakon álló elképzeléseink vannak. Ez az állapot pedig évtizedek óta egyre csak romlik az idő előrehaladtával... Szerencsére ezt nem kell és nem is lehet a tudomány bukásának elkönyvelnünk, pont ellenkezőleg! A '80-as évektől kezdődően a számítástechnika robbanásszerű fejlődésével nagyon rövid idő alatt olyan ajtók nyíltak meg előttünk a világban, amit azelőtt elképzelni sem tudtunk. Ez, annak minden jó és kevésbé jó hozadékával egyetemben, gyökeresen formálta át életünk minden részletét. Ennek a technológiai robbanásnak erőteljes hatása volt többek között a tudományra is. Az exponenciálisan egyre csak növekvő precizitású mérőműszerek egymás utáni, -- addigiakhoz képest -- gyors megjelenése szinte \q{természetessé} vált. Már nem az volt a kérdés, hogy egy érdekes és fontos fizikai jelenséget \textit{képesek leszünk-e valaha} megmérni, hanem hogy az \textit{5, vagy 10 éven belül fog-e sikerülni}?

Ebben az elmúlt kb. 40 évben az egyre pontosabb és pontosabb mérések átformálták a kozmológiáról addig alkotott képünket. Számtalan olyan hibás gondolatra, vagy egyszerűen tátongó lyukakra mutattak rá az addigi modelljeinkben, amik precíziós mérések, jelentős számítási kapacitást igénylő szimulációk és nagyméretű adatok feldolgozását lehetővé tevő módszerek nélkül nem lettek volna sosem észrevehetőek. \textbf{Ide kell még 1-2 átvezető mondat.} Habár jól ismert, hogy ez sokaknak ellentmondásnak hangzik, pontosan ez az, ami annyira biztossá tehet minket a tudomány örökérvényűségében: Hisz hiába bukjon ki egy mindent felforgató újdonság, vagy hiába alakuljon át gyökeresen a véleményünk egy adott gondolatról, az eredmény minden esetben az emberiség előrelépése lesz a világ pontosabb megismerése felé. De ez már egy másik történet...

