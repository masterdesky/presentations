%----------------------------------------------------------------------------------------
%	SLIDE 10.
%----------------------------------------------------------------------------------------
\begin{frame}
\frametitle{Realistic nuclear matter equation of state (EoS)}

\begin{itemize}
	\item The goal of constructing an EoS is to describe the thermodynamical properties (mainly the energy density $\epsilon$ and the pressure $p$) of the nuclear matter that the NS consists of.
	\item The crust and core of the NS usually described using different models.
	\item There is a real abundance of models trying to describe NS in any way. In the modern NS literature, models based on relativistic mean-field theory became the most widely used ones.
	\item In Glendenning's Compact stars there are 17 different EoS mentioned and compared with each other (pp. 296--297, Table 6.3 and 6.4).
\end{itemize}
\begin{alertblock}{Main question}
	\centering	
	What are the real constraints on the EoS?
\end{alertblock}

\end{frame}
