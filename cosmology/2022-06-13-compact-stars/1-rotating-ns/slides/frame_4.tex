%----------------------------------------------------------------------------------------
%	SLIDE 4.
%----------------------------------------------------------------------------------------
\begin{frame}
\frametitle{Dimensional analysis of $\omega (r)$}

\begin{itemize}
	\item Since frame-dragging is caused by the rotation of the star, it will be proportional to the angular momentum. Since we're only interested in dimensions, we can write the classical formula for this:
	\begin{block}{}
		\begin{equation} \label{eq:3}
			J = I * \Omega \propto M R^{2} \Omega
		\end{equation}
	\end{block}
	\item Using CGS units to make dimensional analysis possible (since GR units won't make us any good obviously...), also using CGS, since astronomers likes it much better than SI for some reason
	\begin{block}{}
		\begin{align*}
			&\left[ G \right] = cm^{3} * g^{-1} * s^{-2} = L^{3} * M^{-1} * T^{-2} \\
			&\left[ J \right] = g * cm^{2} * s^{-1} = M * L^{2} * T^{-1} \\
			&\left[ \omega \right] = s^{-1} = T^{-1}
		\end{align*}
	\end{block}
\end{itemize}

\end{frame}