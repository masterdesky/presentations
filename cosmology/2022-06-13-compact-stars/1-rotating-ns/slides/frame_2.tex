%----------------------------------------------------------------------------------------
%	SLIDE 2.
%----------------------------------------------------------------------------------------
\begin{frame}
\frametitle{Frame-dragging}
\framesubtitle{Effect of rotation on metric tensor}

\begin{itemize}
	\item The metric tensor in case of a static, stationary star is described in the Schwarzschield metric with $G = c = 1$ contains only diagonal elements
	\begin{block}{}
		\begin{equation} \label{eq:1}
		\begin{aligned}
			g
			=
			d \tau^{2}
			=
			e^{2 \nu \left( r \right)} dt^{2}
			-
			e^{2 \lambda \left( r \right)} dr^{2}
			-
			r^{2} d \theta^{2}
			-
			r^{2} sin^{2} \left( \theta \right) d \phi^{2}
		\end{aligned}
		\end{equation}
	\end{block}
	\item Rotation however introduces off-diagonal elements too (where $\omega \neq 0$ is the angular velocity of local interial frames)
	\begin{block}{}
		\begin{equation} \label{eq:2}
		\begin{aligned}
			g
			=
			d \tau^{2}
			=&
			\ e^{2 \nu \left( r, \theta \right)} dt^{2}
			-
			e^{2 \lambda \left( r, \theta \right)} dr^{2} 
			\\
			&-
			e^{2 \mu \left( r, \theta \right)}
			\left[
				r^{2} d \theta^{2} + r^{2} \sin^{2} \left( \theta \right)
				*
				\left(
					d \phi - \omega \left( r, \theta \right) dt
				\right)^{2}
			\right]
		\end{aligned}
		\end{equation}
	\end{block}
	\item Due to rotation the star deforms around the equator (centrifugal flattening) and loses spherical symmetry
\end{itemize}

\end{frame}